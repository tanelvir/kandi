\documentclass[finnish]{tktltiki2}

% tktltiki2 automatically loads babel, so you can simply
% give the language parameter (e.g. finnish, swedish, english, british) as
% a parameter for the class: \documentclass[finnish]{tktltiki2}.
% The information on title and abstract is generated automatically depending on
% the language, see below if you need to change any of these manually.
% 
% Class options:
% - grading                 -- Print labels for grading information on the front page.
% - disablelastpagecounter  -- Disables the automatic generation of page number information
%                              in the abstract. See also \numberofpagesinformation{} command below.
%
% The class also respects the following options of article class:
%   10pt, 11pt, 12pt, final, draft, oneside, twoside,
%   openright, openany, onecolumn, twocolumn, leqno, fleqn
%
% The default font size is 11pt. The paper size used is A4, other sizes are not supported.
%
% rubber: module pdftex

% --- General packages ---

\usepackage[utf8]{inputenc}
\usepackage[T1]{fontenc}
\usepackage{lmodern}
\usepackage[pdftex]{graphicx}
\usepackage{subfigure}
\usepackage{microtype}
\usepackage{amsfonts,amsmath,amssymb,amsthm,booktabs,color,enumitem,graphicx}
\usepackage[pdftex,hidelinks]{hyperref}

% Automatically set the PDF metadata fields
\makeatletter
\AtBeginDocument{\hypersetup{pdftitle = {\@title}, pdfauthor = {\@author}}}
\makeatother

% --- Language-related settings ---
%
% these should be modified according to your language

% babelbib for non-english bibliography using bibtex
\usepackage[fixlanguage]{babelbib}
\selectbiblanguage{finnish}

% add bibliography to the table of contents
\usepackage[nottoc]{tocbibind}
% tocbibind renames the bibliography, use the following to change it back
\settocbibname{Lähteet}

% --- Theorem environment definitions ---

\newtheorem{lau}{Lause}
\newtheorem{lem}[lau]{Lemma}
\newtheorem{kor}[lau]{Korollaari}

\theoremstyle{definition}
\newtheorem{maar}[lau]{Määritelmä}
\newtheorem{ong}{Ongelma}
\newtheorem{alg}[lau]{Algoritmi}
\newtheorem{esim}[lau]{Esimerkki}

\theoremstyle{remark}
\newtheorem*{huom}{Huomautus}


% --- tktltiki2 options ---
%
% The following commands define the information used to generate title and
% abstract pages. The following entries should be always specified:

\title{Digitaaliset allekirjoitukset mobiiliympäristössä}
\author{Taneli Virkkala}
\date{\today}
\level{Kandidaatin tutkielma}
\abstract{Tiivistelmä.}

% The following can be used to specify keywords and classification of the paper:

\keywords{avainsana 1, avainsana 2, avainsana 3}

% classification according to ACM Computing Classification System (http://www.acm.org/about/class/)
% This is probably mostly relevant for computer scientists
% uncomment the following; contents of \classification will be printed under the abstract with a title
% "ACM Computing Classification System (CCS):"
% \classification{}

% If the automatic page number counting is not working as desired in your case,
% uncomment the following to manually set the number of pages displayed in the abstract page:
%
% \numberofpagesinformation{16 sivua + 10 sivua liitteissä}
%
% If you are not a computer scientist, you will want to uncomment the following by hand and specify
% your department, faculty and subject by hand:
%
% \faculty{Matemaattis-luonnontieteellinen}
% \department{Tietojenkäsittelytieteen laitos}
% \subject{Tietojenkäsittelytiede}
%
% If you are not from the University of Helsinki, then you will most likely want to set these also:
%
% \university{Helsingin Yliopisto}
% \universitylong{HELSINGIN YLIOPISTO --- HELSINGFORS UNIVERSITET --- UNIVERSITY OF HELSINKI} % displayed on the top of the abstract page
% \city{Helsinki}
%


\begin{document}

% --- Front matter ---

\frontmatter      % roman page numbering for front matter

\maketitle        % title page
\makeabstract     % abstract page

\tableofcontents  % table of contents

% --- Main matter ---

\mainmatter       % clear page, start arabic page numbering

\section{Johdanto}

% Write some science here.

Digitaalisten allekirjoitusten käyttö on noussut huomattavasti mobiililaitteilla nykypäivänä. Mobiiliympäristössä turvallinen yhteys on varmistettava, koska tietoturvariskit langattomissa verkoissa ovat erittäin suuret \cite{enti}. Teknisen kehityksen ansiosta allekirjoitusten luonti mobiililaitteilla on yleistynyt, ja erityistä tietoturvaa vaativat toimenpiteet ovat tulleet mahdollisiksi. PC:llä käytettävät protokollat kuten PKI-malli ovat siirtyneet mobiiliympäristöön sellaisenaan, eivätkä nämä protokollat ole tarvinneet suuria muutoksia toimiakseen. Kehittyneemmän laskentatehon ansiosta monet algoritmit kuten RSA ja Diffie-Hellman ollaan pystytty ottamaan käyttöön kannettavilla laitteilla \cite{enti}. Palvelimille voidaan silti delegoida operaatiot, joita laitteella ei pystytä suorittamaan. Huolimatta siitä tehdäänkö allekirjoitus palvelimella vai asiakkaan laitteessa, tulee allekirjoituksen täyttää kaikki sille asetetut ehdot tietoturvaa koskien. Palvelinpohjaisen allekirjoituksen yleensä luo välissä oleva kirjautumispalvelin eikä lopullinen palveluntarjoaja \cite{proxy}.

Digitaalisten allekirjoitusten käyttö mobiiliympäristössä tulisi olla nopeaa ja turvallista. Monet nykyaikaiset sovellukset voivat vaatia jokaisen viestin lähetyksen yhteydessä uuden allekirjoituksen. Esimerkkinä tästä voisi toimia eräänlainen huutokauppasovellus, jossa jokaisen huudon on oltava kiistaton ja todennettu. Lisäksi viestin sisältämän datan tulee olla yhtenäistä. Varmenteet eivät voi siis kokonaan korvata asiakkaan tunnistamista. Sen sijaan jokainen allekirjoitus tarvitsee varmenteen toimiakseen \cite{proxy}.

Schwabin ja Yangin mukaan mahdollisia tietoturvariskejä mobiiliympäristössä ovat urkinta, mies välissä -hyökkäys, datan muuntaminen, toisena osapuolena esiintyminen ja laitteen kadottaminen \cite{enti}. Näitä riskejä vastaan tulee mobiililaitteen sisäisen toiminnan ja verkkoviestinnän olla turvallista. Jos palvelun tarjoajan ja asiakkaan välissä on välityspalvelin, siihen tulee myös muodostaa luotettava yhteys.
Koska digitaaliset allekirjoitukset perustuvat julkisen avaimen protokollaan, on äärimmäisen tärkeää pitää salainen avain mahdollisimman piilossa. Laitteen SIM-kortti on turvallinen paikka säilyttää salaista avainta, sillä silloin se ei paljastu laitteen käyttöjärjestelmälle. Laitteen prosessorin luoma avain sekä allekirjoitus paljastuvat aina käyttöjärjestelmälle. Sen sijaan prosessorilla laskenta on nopeampaa kuin SIM-kortilla \cite{proxy}. 

Sekä RSA että Diffie-Hellman käyttävät jakojäännösmenetelmää salauksessa.
Diskreetin logaritmin avulla ulkopuolinen tunkeutuja ei voi tietää puuttuvaa alkulukua. Luvun arvaamiseen kuluisi polynomisen ajan verran nykyaikaisilla algoritmeilla. Diffie-Hellmanin algoritmia käytetään julkisen avaimen vaihtoon ja RSA puolestaan perustuu yksityisen avaimen luontiin asymmetrisen salauksen mahdollistamiseksi. Digitaalisessa allekirjoituksessa sekä datan että salauksen tiivisteen tulee olla samat, jotta voidaan varmistaa allekirjoituksen pätevyys. Tiivisteen laskemiseen käytetään erilaisia tiivistefunktioita kuten SHA-2 tai MD5 \cite{gene}.



   
      


\section{Digitaalisen allekirjoituksen määritelmä}

Digitaalinen allekirjoitus on menetelmä, jolla voidaan todentaa tietyn lähettäjän lähettäneen viestin vastaanottajalle muuttumattomana. Allekirjoituksen ja datan tiivisteestä voidaan varmentaa tiedon muuttumattomuus ja todentaa lähettäjän kiistämättömyys \cite{moen}. Jos tieto allekirjoituksessa tai tiivisteessä muuttuu, vastaanottajan avaimella purettu viesti ei ole ymmärrettävässä muodossa enää. Seuraavien ehtojen on oltava voimassa allekirjoituksessa: uskottavuus, muuttumattomuus, kertakäyttöisyys ja kiistattomuus \cite{e-c}. Digitaalisella allekirjoituksella voidaan siis todentaa vain yksi viesti kerrallaan ja jokaiselle viestille on luotava uusi allekirjoitus. 

\subsection{PKI-malli}

Julkinen ja salainen avain muodostavat PKI-mallin. Digitaalinen allekirjoitus perustuu tähän malliin ja siksi salaisen avaimen on pysyttävä vain lähettäjän hallussa. Sen sijaan julkinen avain annetaan vastaanottajalle, joka voi purkaa salatun viestin ja laskea tiivisteet. Koska avainten luomiseen käytetään monimutkaisia algoritmeja kuten RSA, on toisen identtisen avainparin syntyminen erittäin epätodennäköistä. Allekirjoitus voidaan liittää viestiin tai lähettää erillisenä \cite{moen}.

\subsection{RSA}

Menetelmän kehittäjien sukunimien mukaan nimetty RSA on salausalgoritmi, joka jakojäännöksen avulla hoitaa salauksen ja purkamisen. Aluksi valitaan kaksi alkulukua $p$ ja $q$, jotka eivät saa olla samat. Näiden lukujen tulo on $N$, jonka jälkeen valitaan kokonaisluku $e$ väliltä $1 < e < N$.   

\subsection{Mobiilikaupankäynti}

Mobiilikaupankäynnillä tarkoitetaan mobiililaitteella tehtäviä maksutransaktioita tai ostotapahtuman vahvistavia viestejä. Menetelmä on siis osa elektronista kaupankäyntiä, jossa käytetään digitaalisia allekirjoituksia \cite{e-c}. Schwab ja Yang toteavat suurten datamäärien varastoinnin olevan yleistä nykyaikaisilla mobiililaitteilla \cite{enti}. Samadanin, Shajarin ja Ahanihan artikkelissa esitellään huutokauppasovellus, joka vaatii jokaisen huudon varmistuksen lyhyen ajan sisällä \cite{proxy}. Allekirjoitusten luonti tulee olla siis nopeaa mobiililaitteilla tietoturva huomioon ottaen. Sekä laitepohjaisia että palvelinpohjaisia allekirjoituksia käytetään mobiilikaupankäynnissä \cite{proxy}.

\section{Laitepohjaiset allekirjoitukset}

Mobiililaite koostuu SIM-kortista ja laitteesta, jossa allekirjoituksen luonti tapahtuu prosessorilla \cite{proxy}. Tietoturvan kannalta SIM-korttia voidaan pitää parempana vaihtoehtona, mutta nopeudessa allekirjoitusten luontia koskien prosessori on tehokkaampi. Salaisen avaimen säiltyspaikka tulee kuitenkin valita turvallisesti, jotta ulkopuolinen tarkkailija ei saa tietää salaista avainta.
 
\subsection{SIM-kortilta luonti}

Laitteen SIM-korttia voidaan pitää turvallisimpana paikkana säilyttää salaista avainta. Digitaalisen allekirjoituksen luomisessa kuitenkin avain joudutaan hetkellisesti näyttämään laitteen käyttöjärjestelmälle. Pieni tietoturvariski on siis olemassa käyttöjärjestelmää koskien. Lisäksi SIM-kortin laskentakapasiteetti on huomattavasti pienempi kuin laitteen prosessorin. ?????? 

\subsection{Laitteen prosessorilta luonti}

Salaisen avaimen säilytys voi tapahtua myös laitteen muistissa. Digitaalinen allekirjoitus luodaan tällöin laitteen prosessorilla, joka on laskentateholtaan huomattavasti tehokkaampi kuin SIM-kortti. Laitteen käyttöjärjestelmässä voi kuitenkin olla tietoturva-aukko, jota hyväksikäyttäen tunkeutujat voivat saada haltuunsa käyttäjän yksityisen avaimen \cite{proxy}.

\subsection{Tunnistautuminen laitteella}
Kun käyttäjä haluaa lähettää viestin palvelimelle tai toiselle käyttäjälle, on tärkeä suosia turvallista protokollaa.

\section{Palvelinpohjaiset allekirjoitukset}
\subsection{Välityspalvelin}
\subsection{NRS ja NRR}
\subsection{Varmenteet}

\section{Vertailu}
\subsection{Tietoturva}
\subsection{Tehokkuus}
\subsection{Nykyaikaisten menetelmien käyttö}

\section{Yhteenveto}  



% --- References ---
%
% bibtex is used to generate the bibliography. The babplain style
% will generate numeric references (e.g. [1]) appropriate for theoretical
% computer science. If you need alphanumeric references (e.g [Tur90]), use
%
% bibliographystyle{babplain-lf}
%
% instead.

\newpage
\bibliographystyle{babalpha-lf}
\bibliography{viitteet}


% --- Appendices ---

% uncomment the following

%\newpage
%\appendix
 
%\section{Esimerkkiliite}

\end{document}