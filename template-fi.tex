\documentclass[finnish]{tktltiki2}

% tktltiki2 automatically loads babel, so you can simply
% give the language parameter (e.g. finnish, swedish, english, british) as
% a parameter for the class: \documentclass[finnish]{tktltiki2}.
% The information on title and abstract is generated automatically depending on
% the language, see below if you need to change any of these manually.
% 
% Class options:
% - grading                 -- Print labels for grading information on the front page.
% - disablelastpagecounter  -- Disables the automatic generation of page number information
%                              in the abstract. See also \numberofpagesinformation{} command below.
%
% The class also respects the following options of article class:
%   10pt, 11pt, 12pt, final, draft, oneside, twoside,
%   openright, openany, onecolumn, twocolumn, leqno, fleqn
%
% The default font size is 11pt. The paper size used is A4, other sizes are not supported.
%
% rubber: module pdftex

% --- General packages ---

\usepackage[utf8]{inputenc}
\usepackage[T1]{fontenc}
\usepackage{lmodern}
\usepackage[pdftex]{graphicx}
\usepackage{subfigure}
\usepackage{microtype}
\usepackage{amsfonts,amsmath,amssymb,amsthm,booktabs,color,enumitem,graphicx}
\usepackage[pdftex,hidelinks]{hyperref}

% Automatically set the PDF metadata fields
\makeatletter
\AtBeginDocument{\hypersetup{pdftitle = {\@title}, pdfauthor = {\@author}}}
\makeatother

% --- Language-related settings ---
%
% these should be modified according to your language

% babelbib for non-english bibliography using bibtex
\usepackage[fixlanguage]{babelbib}
\selectbiblanguage{finnish}

% add bibliography to the table of contents
\usepackage[nottoc]{tocbibind}
% tocbibind renames the bibliography, use the following to change it back
\settocbibname{Lähteet}

% --- Theorem environment definitions ---

\newtheorem{lau}{Lause}
\newtheorem{lem}[lau]{Lemma}
\newtheorem{kor}[lau]{Korollaari}

\theoremstyle{definition}
\newtheorem{maar}[lau]{Määritelmä}
\newtheorem{ong}{Ongelma}
\newtheorem{alg}[lau]{Algoritmi}
\newtheorem{esim}[lau]{Esimerkki}

\theoremstyle{remark}
\newtheorem*{huom}{Huomautus}


% --- tktltiki2 options ---
%
% The following commands define the information used to generate title and
% abstract pages. The following entries should be always specified:

\title{Digitaaliset allekirjoitukset mobiiliympäristössä}
\author{Taneli Virkkala}
\date{\today}
\level{Kandidaatin tutkielma}
\abstract{Tiivistelmä.}

% The following can be used to specify keywords and classification of the paper:

\keywords{avainsana 1, avainsana 2, avainsana 3}

% classification according to ACM Computing Classification System (http://www.acm.org/about/class/)
% This is probably mostly relevant for computer scientists
% uncomment the following; contents of \classification will be printed under the abstract with a title
% "ACM Computing Classification System (CCS):"
% \classification{}

% If the automatic page number counting is not working as desired in your case,
% uncomment the following to manually set the number of pages displayed in the abstract page:
%
% \numberofpagesinformation{16 sivua + 10 sivua liitteissä}
%
% If you are not a computer scientist, you will want to uncomment the following by hand and specify
% your department, faculty and subject by hand:
%
% \faculty{Matemaattis-luonnontieteellinen}
% \department{Tietojenkäsittelytieteen laitos}
% \subject{Tietojenkäsittelytiede}
%
% If you are not from the University of Helsinki, then you will most likely want to set these also:
%
% \university{Helsingin Yliopisto}
% \universitylong{HELSINGIN YLIOPISTO --- HELSINGFORS UNIVERSITET --- UNIVERSITY OF HELSINKI} % displayed on the top of the abstract page
% \city{Helsinki}
%


\begin{document}

% --- Front matter ---

\frontmatter      % roman page numbering for front matter

\maketitle        % title page
\makeabstract     % abstract page

\tableofcontents  % table of contents

% --- Main matter ---

\mainmatter       % clear page, start arabic page numbering

\section{Johdanto}

% Write some science here.

Digitaalisten allekirjoitusten käyttö on noussut huomattavasti mobiililaitteilla nykypäivänä. Mobiiliympäristössä turvallinen yhteys on varmistettava, koska tietoturvariskit langattomissa verkoissa ovat erittäin suuret \cite{enti}. Teknisen kehityksen ansiosta allekirjoitusten luonti mobiililaitteilla on yleistynyt, ja erityistä tietoturvaa vaativat toimenpiteet ovat tulleet mahdollisiksi. PC:llä käytettävät protokollat kuten PKI-malli ovat siirtyneet mobiiliympäristöön sellaisenaan, eivätkä nämä protokollat ole tarvinneet suuria muutoksia toimiakseen. Kehittyneemmän laskentatehon ansiosta monet algoritmit kuten RSA ja Diffie-Hellman ollaan pystytty ottamaan käyttöön kannettavilla laitteilla \cite{enti}. Palvelimille voidaan silti delegoida operaatiot, joita laitteella ei pystytä suorittamaan. Huolimatta siitä tehdäänkö allekirjoitus palvelimella vai asiakkaan laitteessa, tulee allekirjoituksen täyttää kaikki sille asetetut ehdot tietoturvaa koskien. Palvelinpohjaisen allekirjoituksen yleensä luo välissä oleva kirjautumispalvelin eikä lopullinen palveluntarjoaja \cite{proxy}.

Digitaalisten allekirjoitusten käyttö mobiiliympäristössä tulisi olla nopeaa ja turvallista. Monet nykyaikaiset sovellukset voivat vaatia jokaisen viestin lähetyksen yhteydessä uuden allekirjoituksen. Esimerkkinä tästä voisi toimia eräänlainen huutokauppasovellus, jossa jokaisen huudon on oltava kiistaton ja todennettu. Lisäksi viestin sisältämän datan tulee olla yhtenäistä. Varmenteet eivät voi siis kokonaan korvata asiakkaan tunnistamista. Sen sijaan jokainen allekirjoitus tarvitsee varmenteen toimiakseen \cite{proxy}.

Schwabin ja Yangin mukaan mahdollisia tietoturvariskejä mobiiliympäristössä ovat urkinta, mies välissä -hyökkäys, datan muuntaminen, toisena osapuolena esiintyminen ja laitteen kadottaminen \cite{enti}. Näitä riskejä vastaan tulee mobiililaitteen sisäisen toiminnan ja verkkoviestinnän olla turvallista. Jos palvelun tarjoajan ja asiakkaan välissä on välityspalvelin, siihen tulee myös muodostaa luotettava yhteys.
Koska digitaaliset allekirjoitukset perustuvat julkisen avaimen protokollaan, on äärimmäisen tärkeää pitää salainen avain mahdollisimman piilossa. Laitteen SIM-kortti on turvallinen paikka säilyttää salaista avainta, sillä silloin se ei paljastu laitteen käyttöjärjestelmälle. Laitteen prosessorin luoma avain sekä allekirjoitus paljastuvat aina käyttöjärjestelmälle. Sen sijaan prosessorilla laskenta on nopeampaa kuin SIM-kortilla \cite{proxy}. 

Sekä RSA että Diffie-Hellman käyttävät jakojäännösmenetelmää salauksessa.
Diskreetin logaritmin avulla ulkopuolinen tunkeutuja ei voi tietää puuttuvaa alkulukua. Luvun arvaamiseen kuluisi polynomisen ajan verran nykyaikaisilla algoritmeilla. Diffie-Hellmanin algoritmia käytetään julkisen avaimen vaihtoon ja RSA puolestaan perustuu yksityisen avaimen luontiin asymmetrisen salauksen mahdollistamiseksi. Digitaalisessa allekirjoituksessa sekä datan että salauksen tiivisteen tulee olla samat, jotta voidaan varmistaa allekirjoituksen pätevyys. Tiivisteen laskemiseen käytetään erilaisia tiivistefunktioita kuten SHA-2 tai MD5 \cite{gene}.         


\section{Digitaalisen allekirjoituksen määritelmä}

Digitaalinen allekirjoitus on menetelmä, jolla voidaan todentaa tietyn lähettäjän lähettäneen viestin vastaanottajalle muuttumattomana. Allekirjoituksen ja datan tiivisteestä voidaan todentaa tiedon muuttumattomuus ja  lähettäjän kiistämättömyys \cite{moen}. Jos tieto allekirjoituksessa tai tiivisteessä muuttuu, vastaanottajan avaimella purettu viesti ei ole ymmärrettävässä muodossa enää. Seuraavien ehtojen on oltava voimassa allekirjoituksessa: uskottavuus, muuttumattomuus, kertakäyttöisyys ja kiistattomuus \cite{e-c}. Digitaalisella allekirjoituksella voidaan siis todentaa vain yksi viesti kerrallaan ja jokaiselle viestille on luotava uusi allekirjoitus. Menetelmä on yksi turvallisimmista tavoista varmentaa luotettava viestinkulku vastaanottajan ja lähettäjän välillä. Sen sijaan allekirjoitus on raskasta luoda, joten menetelmä vaatii merkittävää laskentatehoa toimiakseen \cite{proxy}.

\subsection{PKI-malli}

Julkinen ja salainen avain muodostavat PKI-mallin \cite{ECC}. Digitaalinen allekirjoitus perustuu tähän malliin ja siksi salaisen avaimen on pysyttävä vain lähettäjän hallussa. Sen sijaan julkinen avain annetaan vastaanottajalle, joka voi purkaa salatun viestin ja laskea tiivisteet. Koska avainten luomiseen käytetään monimutkaisia algoritmeja kuten RSA, on toisen identtisen avainparin syntyminen erittäin epätodennäköistä. Allekirjoitus voidaan liittää viestiin tai lähettää erillisenä \cite{moen}. Hyvänä pituutena tietoturvan kannalta molemmille avaimille voidaan pitää vähintään 1024 bittiä \cite{ECC}. 

\subsection{RSA}

Menetelmän kehittäjien sukunimien mukaan nimetty RSA on salausalgoritmi, joka jakojäännöksen avulla hoitaa salauksen ja purkamisen. Aluksi valitaan kaksi alkulukua $p$ ja $q$, jotka eivät saa olla samat. Näiden lukujen tulo on $N$, jonka jälkeen valitaan kokonaisluku $e$ väliltä $1 < e < N$. (jatkuu)

\subsection{Mobiilikaupankäynti}

Mobiilikaupankäynnillä tarkoitetaan mobiililaitteella tehtäviä maksutransaktioita tai ostotapahtuman vahvistavia viestejä. Menetelmä on siis osa elektronista kaupankäyntiä, jossa käytetään digitaalisia allekirjoituksia \cite{e-c}. Schwab ja Yang toteavat suurten datamäärien varastoinnin olevan yleistä nykyaikaisilla mobiililaitteilla \cite{enti}. Samadanin, Shajarin ja Ahanihan artikkelissa esitellään huutokauppasovellus, joka vaatii jokaisen huudon varmistuksen lyhyen ajan sisällä \cite{proxy}. Allekirjoitusten luonti tulee olla siis nopeaa mobiililaitteilla tietoturva huomioon ottaen. Sekä laitepohjaisia että palvelinpohjaisia allekirjoituksia käytetään mobiilikaupankäynnissä \cite{proxy}. Nykykään myös verkkopankkisovelluksia voi käyttää mobiilimuodossa.

\section{Laitepohjaiset allekirjoitukset}

Mobiililaite koostuu SIM-kortista ja laitteesta, jossa allekirjoituksen luonti tapahtuu prosessorilla. Tietoturvan kannalta SIM-korttia voidaan pitää parempana vaihtoehtona, mutta allekirjoitusten luomisen nopeudessa prosessori on tehokkaampi. Salaisen avaimen säilytyspaikka tulee kuitenkin valita turvallisesti, jotta ulkopuolinen tarkkailija ei saa tietää salaista avainta. Lisäksi on olemassa malli, jossa SIM-kortti ja laitteen prosessori yhdessä osallistuvat allekirjoituksen luontiin (hybridimalli) \cite{proxy}. Seuraava jaottelu perustuu Samadanin, Shajarin ja Ahanihan malleihin. 
 
\subsection{SIM-kortilta luonti}

Laitteen SIM-korttia voidaan pitää turvallisimpana paikkana säilyttää salaista avainta. Edes käyttäjä itse tai laitteen käyttöjärjestelmä eivät pääse käsiksi salaiseen avaimeen kortilla. Kuitenkin SIM-kortin laskentakapasiteetti on huomattavasti pienempi kuin laitteen prosessorin. Allekirjoituksen luonti siis SIM-kortilta on erittäin hidasta \cite{proxy}.  

\subsection{Laitteen prosessorilta luonti}

Salaisen avaimen säilytys voi tapahtua myös laitteen muistissa. Digitaalinen allekirjoitus luodaan tällöin laitteen prosessorilla, joka on laskentateholtaan huomattavasti tehokkaampi kuin SIM-kortti. Käyttöjärjestelmä voi myös tarjota kirjastoja ja työkaluja allekirjoitusten luontiin. Laitteen käyttöjärjestelmässä voi kuitenkin olla tietoturva-aukko, jota hyväksikäyttäen tunkeutujat voivat saada haltuunsa käyttäjän yksityisen avaimen \cite{proxy}.

\subsection{Hybridimalli}

Hybridimallissa salainen avain joudutaan hetkellisesti paljastamaan laitteen käyttöjärjestelmälle. Pieni tietoturvariski on siis olemassa tässä menetelmässä. Hyvänä puolena hybridimallissa on sen lähes yhtä nopea tehokkuus kuin prosessorilta luonnissa. Monet graafisen käyttöliitymän vaativat ohjelmat tarvitsevat prosessorin laskentatehoa, mutta SIM-kortti voi toimia tietoturvan kannalta avaimen yleisenä säilytyspaikkana. Mallissa allekirjoitus siis luodaan prosessorilla, jolloin salaista avainta käytetään vain hetkellisesti laitteessa \cite{proxy}.



\subsection{Tunnistautuminen laitteella}
Kun käyttäjä haluaa lähettää viestin palvelimelle tai toiselle käyttäjälle, on tärkeä suosia turvallista protokollaa. On turvallista varmistaa myös oikean henkilön käyttävän laitetta, sillä ulkopuolinen varas on voinut anastaa laitteen. Käyttäjän tunnistautuminen voi perustua salasanan syöttämiseen tai visuaaliseen todennukseen. Istunto laitteen ja palvelimen välille voidaan muodostaa Diffie-Hellman protokollaa käyttäen. Yhteisellä avaimella siis hoidetaan viestien salaus. RSA on kuitenkin parempi mies välissä- hyökkäystä vastaan \cite{enti}.

\section{Palvelinpohjaiset allekirjoitukset}

Palvelin voi luoda digitaalisen allekirjoituksen käyttäjän puolesta, kunhan käyttäjä voidaan todentaa palvelimelle. Palvelinten rooli digitaalisten allekirjoitusten luonnissa oli merkittävä aikana, jolloin laitteessa ei ollut tarpeeksi tehoa 
allekirjoituksen luomiseen. Nykyään laitepohjaiset allerkijoitukset ovat yleistyneet \cite{proxy}.

\subsection{Välityspalvelin}

Välityspalvelin toimii siis eräänlaisena kirjautumispalvelimena käyttäjän ja lopullisen palveluntarjoajan välissä. Välityspalvelin voi luoda allekirjoituksen, mutta oikean käyttäjän varmenne vaaditaan. Varmennus voi perustua algoritmeihin kuten RSA tai DSA. On myös mahdollista, että käyttäjälle tehdään varmenne, jolla hän on tunnistettavissa jatkossa palvelimelle \cite{proxy}.

\subsection{NRS ja NRR}

Kiistämättömyys on olennainen osa digitaalista allekirjoitusta. NRS (Non-Repudation of Sender) tarkoittaa, lähettäjä ei voi jälkikäteen kiistää lähettäneensä viestin. NRR (Non-Repudation of Receiver) puolestaan merkitsee vastaanottajan kiistämättömyyttä. Tiivistefunktioilla varmistetaan datan eheys kuten esimerkiksi MD5:llä. Sekä lähettäjän että vastaanottajan on luotava julkiset avaimet ja merkit kirjautumispalvelimelle tunnistettavaksi. Kirjautumispalvelin pyytää varmenteen varmenneviranomaiselta ja muodostaa oman varmenteen lähettäjälle. Näin ollen kirjautumispalvelin voi jatkossa toimia pysyvämpänä vahvistajana linkkinä lähettäjän ja vastaanottajan välillä \cite{gene}.


\subsection{Varmenteet}

Varmenteet ovat tapa tunnistaa jokin käyttäjä, välityspalvelin tai lopullinen palveluntarjoaja jatkuvaa yhteydenpitoa varten. Varmenne voi olla voimassa hetken tai pidemmän aikaa, mutta tietoturvan kannalta varmenteiden ei tulisi olla ikuisia. Varmennetta voidaan pitää luotettavana, jos sen tarjoaa ulkopuolinen varmenneviranomainen. Digitaalinen allekirjoitus vaatii toimiakseen aina varmenteen, mutta varmenne voi toimia irrallisena digitaalisesta allekirjoituksesta. PKI-protokollan avulla varmenne voidaan luoda luovuttamalla julkinen avain varmenneviranomaiselle ja lähettämällä varmennepyyntö. Tämän jälkeen käyttäjä vahvistaa vielä itsensä salaamalla viestinsä yksityisellä avaimellaan. Varmenneviranomainen vastaa luovuttamalla varmenteen käyttäjälle. Varmenteeseen on yleensä merkitty seuraavat tiedot: voimassaoloaika, sarjanumero, versio ja käyttäjän tunniste \cite{ECC}. Vastaanottajan tulee siis ottaa huomioon vanhentunut varmenne. Koska varmenne on käyttäjäkohtainen, hyökkääjä ei tee varastetulla varmenteella mitään. 

\section{Vertailu}

Tehokkuus ja tietoturva ovat tärkeitä ominaisuuksia koskien digitaalisia allekirjoituksia. Vaikka nämä kaksi seikkaa eivät ole suoraan toisensa poissulkevia, on syytä ottaa huomioon kummankin prioriteetti. Erityisesti mobiililaitteilla tehokkuudesta joudutaan yleensä karsimaan, joten allekirjoituksen luonti voi viedä huomattavan ajan \cite{proxy}. 

\subsection{Tietoturva}

Mobiililaitteilla voidaan havaita seuraavia tietoturvariskejä: urkinta, mies välissä -hyökkäys, datan muuntaminen, toisena osapuolena esiintyminen ja laitteen kadottaminen. Urkinnalla tarkoitetaan viestien kuuntelua, mutta se voidaan torjua helposti viestin salakirjoituksella esimerkiksi väliaikaisella istuntoavaimella. Mies välissä- hyökkäys tarkoittaa kolmannen osapuolen asettumista lähettävän ja vastaanottavan osapuolten väliin. Diffie-Hellmanissa piilee tämä riski mutta ei yleensä RSA:ssa. Datan muuntaminen voidaan estää salakirjoituksella sekä käyttämällä tiivistefunktioita. Toisena osapuolena tekeytyminen ja laitteen kadottaminen voidaan estää salasanan kirjoittamisella laitteelle tai visuaalisena todennuksella. 

Julkisen avaimen protokolla eli PKI-malli toimii, jos salainen avain säilyy suojassa. Mikäli on pienikin riski, että salainen avain on jokun muun tiedossa tulee avainpari vaihtaa heti. Niin kauan kun diskreetin logaritmin ongelmaa ei pystytä ratkaisemaan järkevässä ajassa, ovat RSA ja Diffie-Hellman turvallisia protokollia. \cite{enti}.
	  

\subsection{Tehokkuus}

Suorituskyky on parantunut vuosien saatossa niin PC- kuin mobiililaitteilla. Prosessorien teknologia on kehittynyt mahdollistaen tiheämmät kellopulssit ja moniydinsuorituksen. Myös tietoliikennenopeuksien nousulla on ollut suuri merkitys digitaalisten allekirjoitusten luonnissa. Tehokkuutta tarvitaan nopeisiin allekirjoituksiin lyhyellä aikavälillä. Artikkelissa Self-Proxy Mobile Signature esitelty huutokauppasovellus tarvitsee jokaiselle huudolle uuden  allekirjoituksen lyhyen ajan sisällä. Tietoturvasta on tässä tapauksessa erittäin vaikea tinkiä, joten käyttäjä olisi hyvä luoda allekirjoitus omalta laitteeltaan. Tehokkuudessa tulee ottaa huomioon siis salauksen nopeus, tiivisteen luominen ja varmenteen hankinta \cite{proxy}. Luonnollisesti myös palvelinpuolella esimerkiksi klusterointi on luonut mahdollisuuden tehokkaisiin allekirjoitusten/varmenteiden luomiseen monelle käyttäjälle samaan aikaan.

\subsection{Nykyaikaisten menetelmien käyttö}

Laitepohjaiset allekirjoitukset ovat yleistynet kokoajan mobiililaitteiden laskentatehon ansiosta. RSA:n lisäksi elliptiset käyrät ovat yleistyneet niiden paremman tietoturvan ansiosta suhteessa avainten pituuteen bitteinä \cite{ECC}. AES algoritmia voidaan pitää murtumattomana, mutta DSA on murrettavissa jo muutaman bittivuodon avulla \cite{gsm}. Elliptisen käyrän DSA:ta käytetään myös mobiililaitteilla \cite{webs}. RSA:n avaimen pituuden on hyvä olla vähintään 1024 bittiä. Elliptisissä käyrissä riittää 160 bittiä tällä hetkellä \cite{ECC}.   

\section{Yhteenveto}  

Tässä tekstissä olemme tarkastelleet digitaalisia allekirjoituksia mobiiliympäristöissä ja mobiililaitteissa. Menetelmät allekirjoitusten luontiin siis vastaavat tietokoneilla samanlaisia menetelmiä. Olemma tarkastelleet PKI-mallia ja RSA algoritmia tarkemmin sekä mobiilikaupankäyntiä. Digitaalisten allekirjoitusten luonti voidaan jakaa kahteen pääryhmään: laite- ja palvelinpohjaisiin allekirjoituksiin. Laiteella on allekirjoituksen voi luoda prosessori tai SIM-kortti. Lisäksi hybridimallin olemassaolo tunnetaan. Palvelinpuolella tulee korostua käyttäjän tunnistaminen ja kirjautumispalvelimen merkitys. Varmenteet ja kiistattomuus luovat digitaalisen allekirjoituksen pohjan. Olemme tarkastelleet tekstin lopussa tietoturvan ja tehokkuuden merkitystä digitaalisissa allekirjoituksissa mobiiliympäristössä. Nykyaikaisiin menetelmiin voimme luetella RSA:n, DSA:n, Diffie-Hellmanin ja elliptisten käyrien algoritmit.


% --- References ---
%
% bibtex is used to generate the bibliography. The babplain style
% will generate numeric references (e.g. [1]) appropriate for theoretical
% computer science. If you need alphanumeric references (e.g [Tur90]), use
%
% bibliographystyle{babplain-lf}
%
% instead.

\newpage
\bibliographystyle{babalpha-lf}
\bibliography{viitteet}


% --- Appendices ---

% uncomment the following

%\newpage
%\appendix
 
%\section{Esimerkkiliite}

\end{document}