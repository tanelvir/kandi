\documentclass[finnish]{tktltiki2}

% tktltiki2 automatically loads babel, so you can simply
% give the language parameter (e.g. finnish, swedish, english, british) as
% a parameter for the class: \documentclass[finnish]{tktltiki2}.
% The information on title and abstract is generated automatically depending on
% the language, see below if you need to change any of these manually.
% 
% Class options:
% - grading                 -- Print labels for grading information on the front page.
% - disablelastpagecounter  -- Disables the automatic generation of page number information
%                              in the abstract. See also \numberofpagesinformation{} command below.
%
% The class also respects the following options of article class:
%   10pt, 11pt, 12pt, final, draft, oneside, twoside,
%   openright, openany, onecolumn, twocolumn, leqno, fleqn
%
% The default font size is 11pt. The paper size used is A4, other sizes are not supported.
%
% rubber: module pdftex

% --- General packages ---

\usepackage[utf8]{inputenc}
\usepackage[T1]{fontenc}
\usepackage{lmodern}
\usepackage[pdftex]{graphicx}
\usepackage{subfigure}
\usepackage{microtype}
\usepackage{amsfonts,amsmath,amssymb,amsthm,booktabs,color,enumitem,graphicx}
\usepackage[pdftex,hidelinks]{hyperref}

% Automatically set the PDF metadata fields
\makeatletter
\AtBeginDocument{\hypersetup{pdftitle = {\@title}, pdfauthor = {\@author}}}
\makeatother

% --- Language-related settings ---
%
% these should be modified according to your language

% babelbib for non-english bibliography using bibtex
\usepackage[fixlanguage]{babelbib}
\selectbiblanguage{finnish}

% add bibliography to the table of contents
\usepackage[nottoc]{tocbibind}
% tocbibind renames the bibliography, use the following to change it back
\settocbibname{Lähteet}

% --- Theorem environment definitions ---

\newtheorem{lau}{Lause}
\newtheorem{lem}[lau]{Lemma}
\newtheorem{kor}[lau]{Korollaari}

\theoremstyle{definition}
\newtheorem{maar}[lau]{Määritelmä}
\newtheorem{ong}{Ongelma}
\newtheorem{alg}[lau]{Algoritmi}
\newtheorem{esim}[lau]{Esimerkki}

\theoremstyle{remark}
\newtheorem*{huom}{Huomautus}


% --- tktltiki2 options ---
%
% The following commands define the information used to generate title and
% abstract pages. The following entries should be always specified:

\title{Digitaaliset allekirjoitukset mobiiliympäristössä}
\author{Taneli Virkkala}
\date{\today}
\level{Kandidaatin tutkielma}
\abstract{Tiivistelmä.}

% The following can be used to specify keywords and classification of the paper:

\keywords{avainsana 1, avainsana 2, avainsana 3}

% classification according to ACM Computing Classification System (http://www.acm.org/about/class/)
% This is probably mostly relevant for computer scientists
% uncomment the following; contents of \classification will be printed under the abstract with a title
% "ACM Computing Classification System (CCS):"
% \classification{}

% If the automatic page number counting is not working as desired in your case,
% uncomment the following to manually set the number of pages displayed in the abstract page:
%
% \numberofpagesinformation{16 sivua + 10 sivua liitteissä}
%
% If you are not a computer scientist, you will want to uncomment the following by hand and specify
% your department, faculty and subject by hand:
%
% \faculty{Matemaattis-luonnontieteellinen}
% \department{Tietojenkäsittelytieteen laitos}
% \subject{Tietojenkäsittelytiede}
%
% If you are not from the University of Helsinki, then you will most likely want to set these also:
%
% \university{Helsingin Yliopisto}
% \universitylong{HELSINGIN YLIOPISTO --- HELSINGFORS UNIVERSITET --- UNIVERSITY OF HELSINKI} % displayed on the top of the abstract page
% \city{Helsinki}
%


\begin{document}

% --- Front matter ---

\frontmatter      % roman page numbering for front matter

\maketitle        % title page
\makeabstract     % abstract page

\tableofcontents  % table of contents

% --- Main matter ---

\mainmatter       % clear page, start arabic page numbering

\section{Johdanto}

% Write some science here.

Digitaalisten allekirjoitusten käyttö on noussut huomattavasti mobiililaitteilla nykypäivänä. Mobiiliympäristössä turvallinen yhteys on varmistettava, koska tietoturvariskit langattomissa verkoissa ovat erittäin suuret \cite{enti}. Teknisen kehityksen ansiosta allekirjoituksia voidaan luoda yleisesti mobiililaitteilla ja parempi tietoturva on mahdollistanut useiden sovellusten käytön. Tietokonelaitteistolla käytettävät protokollat kuten esimerkiksi PKI-malli (julkisen avaimen infrastruktuuri) ovat siirtyneet mobiiliympäristöön sellaisenaan, eivätkä nämä protokollat ole tarvinneet suuria muutoksia toimiakseen. Kehittyneemmän laskentatehon ansiosta monet algoritmit kuten RSA ja Diffie-Hellman ollaan pystytty ottamaan käyttöön kannettavilla laitteilla \cite{enti}. Palvelimille voidaan silti delegoida operaatiot, joita laitteella ei pystytä suorittamaan. Allekirjoitus voidaan luoda tarvittaessa palvelimella tai asiakkaan laitteessa, mutta allekirjoituksen tulee täyttää kaikki sille asetetut ehdot tietoturvaa koskien. Palvelinpohjaisen allekirjoituksen yleensä luo välissä oleva kirjautumispalvelin eikä lopullinen palveluntarjoaja \cite{proxy}.

Digitaalisten allekirjoitusten käyttö mobiiliympäristössä tulisi olla nopeaa ja turvallista. Monet nykyaikaiset sovellukset vaativat jokaisen viestin lähetyksen yhteydessä uuden allekirjoituksen. Esimerkkinä tästä voisi toimia eräänlainen huutokauppasovellus, jossa jokaisen huudon on oltava kiistaton ja todennettu. Lisäksi viestin sisältämän datan tulee olla eheää. Varmenne on varmenneviranomaisen tarjoama osa digitaalisen allekirjoituksen luontiin. Jokainen allekirjoitus tarvitsee varmenteen toimiakseen. \cite{proxy}

Schwabin ja Yangin mukaan \cite{enti} mahdollisia tietoturvariskejä mobiiliympäristössä ovat urkinta, välimieshyökkäys, datan muuntaminen, toisena osapuolena esiintyminen ja laitteen kadottaminen. Mobiililaitteen sisäisen toiminnan ja verkkoviestinnän tulee olla suojattu mahdollisilta riskeiltä. Jos palvelun tarjoajan ja asiakkaan välissä on välityspalvelin, siihen tulee myös muodostaa luotettava yhteys. Koska digitaaliset allekirjoitukset perustuvat julkisen avaimen infrastruktuuriin, on äärimmäisen tärkeää pitää salainen avain mahdollisimman turvassa. Laitteen SIM-kortti on turvallinen paikka säilyttää salaista avainta, joka ei silloin paljastu laitteen käyttöjärjestelmälle. Prosessorin luoma avain paljastuu aina kokonaan laitteelle. Sen sijaan prosessorilla avaimen ja allekirjoituksen luonti on nopeampaa kuin SIM-kortilla \cite{proxy}. 

Varmenneviranomaisen myöntämä varmenne on voimassa vain tietyn aikaa. Vastaanottajan on tunnettava kyseinen varmenneviranomainen viestin kelpoisuuden varmistamiseksi. Viestin lähettäjä liittää varmenteen allekirjoitukseen. \cite{ECC}       


\section{Digitaalisen allekirjoituksen määritelmä}

Digitaalinen allekirjoitus on menetelmä, jolla voidaan todentaa tietyn lähettäjän lähettäneen viestin vastaanottajalle muuttumattomana. Lähettäjä ei voi siis jälkikäteen kiistää lähettämäänsä viestiä (kiistämättömyys). Viesti on salattu käyttäen jotain salausmenetelmää kuten esimerkiksi AES-lohkosalausta. Allekirjoitus vaatii toimiakseen tiivistefunktioita ja varmenteen varmenneviranomaiselta. Seuraavassa kappaleessa esiteltyä julkisen avaimen infrastruktuuria käytetään digitaalisissa allekirjoituksissa. Viesti on siis saapunut oikeana perille, mikäli sanoman tiiviste ja allekirjoitettu tiiviste ovat samat luvut.

\subsection{Toiminta} 

Tiivisteistä ja datan eheydestä voidaan todentaa tiedon muuttumattomuus ja lähettäjän kiistämättömyys \cite{moen}. Vastanottaja purkaa viestin allekirjoituksen allekirjoittajan lähettämällä julkisella avaimella eli laskee sanoman tiivisteen ja allekirjoitetun tiivisteen arvon. Molempien arvojen tulisi olla samat, jotta viesti voidaan hyväksyä. Jos viestin sisältöä muutetaan, vastaanottajan avaimella purettu viesti ei ole enää ymmärrettävässä muodossa. Tiivistefunktioita voivat olla SHA-2 ja MD5. On syytä huomata, että SHA-1 ja MD5 ovat jo vanhentuneita tietoturvan kannalta ja uusi standardi SHA-3 on tuloillaan \cite{nist}.   

Viestin varmenteessa on mukana lähettäjän liittämä julkinen avain, jollei sitä ole lähetetty aiemmin. Varmenteen kelpoisuus on varmistettava varmenneviranomaiselta. Lisäksi varmenteen voimassaolo on tarkistettava. Varmenneviranomaisen allekirjoitus on siis tarkastettava viranomaisen julkisella avaimella, joka on saatettu lähettää saman viestin mukana. Seuraavien ehtojen on oltava voimassa allekirjoituksessa: uskottavuus, muuttumattomuus, kertakäyttöisyys ja kiistattomuus \cite{e-c}. Digitaalisella allekirjoituksella voidaan siis todentaa vain yksi viesti kerrallaan, ja jokaiselle viestille on luotava uusi allekirjoitus. Menetelmä on yksi turvallisimmista tavoista varmentaa luotettava viestinkulku vastaanottajan ja lähettäjän välillä. Sen sijaan allekirjoitus on raskasta luoda, joten menetelmä vaatii merkittävää laskentatehoa toimiakseen \cite{proxy}. Erityisen vaikeaksi tekee tilanne, jossa salausta joudutaan käyttämään joka palvelimelle siirryttäessä. Digitaalisia allekirjoituksia käytetään siis yleensä yhteydenmuodostuksen aluksi, mutta niistä voidaan luopua myöhemmissä viestien lähteyksissä.

AES ja 3DES ovat lohkosalausalgoritmeja, jotka salaavat selväkielisen tekstin 
kontekstittomaan muotoon. Ne kummatkin ovat voimassa vielä nykyipäivän tietoturvastandardeissa ja ne soveltuvat esimerkiksi tekstiviestien (SMS) salaukseen. Saxenan ja Chaudharin tutkimuksessa AES oli ajallisesti ja salauksen lopputulokseltaan paras vaihtoehto tekstiviestin salauksessa. Myös purkamisessa AES oli parempi verrattuna 3DES:sään. \cite{gsm}


\subsection{Julkisen avaimen infrastruktuuri}

Julkinen ja salainen avain muodostavat PKI-mallin \cite{ECC}. Digitaalinen allekirjoitus perustuu julkisen avaimen infrastruktuuriin ja siksi salaisen avaimen on pysyttävä vain lähettäjän hallussa. Sen sijaan julkinen avain annetaan vastaanottajalle, joka voi laskea salatun tiivisteen. Kyseessä on siis epäsymmetrinen salaus. Koska avaimia luodaan valtavia määriä perustuen suuriin alkulukuihin RSA:ssa, on toisen identtisen avainparin syntyminen erittäin epätodennäköistä. Allekirjoitus voidaan liittää viestiin tai lähettää erillisenä \cite{moen}. Hyvänä pituutena tietoturvan kannalta molemmille avaimille voidaan pitää vähintään 1024 bittiä \cite{ECC}. 

\subsection{RSA}

Kehittäjien sukunimien mukaan nimetty RSA on salausalgoritmi, joka jakojäännöksen avulla hoitaa salauksen ja purkamisen. Aluksi valitaan kaksi alkulukua $p$ ja $q$, jotka eivät saa olla samat. Näiden lukujen tulo on $n$. Eulerin funktiolla $(p-1)(q-1) = \phi(n)$ saadaan positiivisten kokonaislukujen $n$ määrä osaksi julkista ja salaista avainta. Tämän jälkeen valitaan kokonaisluku $e$ väliltä $1 < e < \phi(n)$. Lisäksi $e$ on suhteellinen alkuluku luvulle $\phi(n)$. Vielä tulee valita luku $d$, joka kerrottuna $e$ jälkeen vähennettynä 1 on jaollinen$\mod{\phi(n)}$. Eli siis kongruenssirelaatiolla $d e \equiv 1\pmod{\phi(n)}$. Tämä voidaan muuttaa muotoon $de \mod{\phi(n)} = 1$. Julkinen avain on pari $(n, e)$ ja salainen avainpari on $(n, d)$.   

Viesti $M$ salataan seuraavalla kaavalla: $$E = M^e \pmod{n}$$

Vastaanottajalla on salainen avain $d$. Purkaminen tapahtuu kaavalla: 
$$ M = E^d \pmod{n} $$

Luvut $n, e$ ja $d$ eivät saa olla pääteltävissä toisistaan. Diskreettinen logaritmi mahdollistaa sen, että ilman puuttuvaa tekijää ei voi päätellä toista. Edellä esitellyt kaavat ovat haettu Wolfram Mathworldistä. \cite{math1}


\subsection{Diffie-Hellman}

Aivan RSA:n tavoin Diffie-Hellman algoritmi käyttää modulaariaritmetiikkaa avainten luonnissa. Diffie-Hellman perustuu yhteen yhteiseen avaimeen, jolla viestittävät osapuolet voivat salata ja purkaa viestit toisillensa. Käyttötarkoitus on vain erilainen. RSA:ta käytetään digitaalisen allekirjoituksen luontiin, mutta Diffie-Hellman on tarkoitettu käyttäjien tunnistamiseen aluksi. Kyseessä on siis symmetrinen salaustyyppi. Diffie-Hellman tietoturvaltaan on altis välimieshyökkäykselle, sillä kolmas osapuoli on voinut saada yhteisen ennalta sovitun avaimen haltuunsa. Protokolla toimii yksinkertaisuudessaan seuraavalla tavalla Wolframin mukaan \cite{math2}.

Aloittaja A ja vastaaja V sopivat etukäteen luvuista $p$ ja $g$. Alkuluku  $p$ ja sen primitiivinen alkio $g$ tulee olla molempien osapuolten tiedossa. Näiden lisäksi viestinvaihdon aloittavan osapuolen A on valittava salainen kokonaisluku $a$. Aloittaja lähettää vastaanottajalle V viestin $A = g^a \pmod{p}$. Tämän jälkeen V valitsee kokonaisluvun $b$, joka myös säilyy salaisena. V lähettää A:lle viestissä luvun $B = g^b \pmod{p}$. A laskee luvun $(g^b \pmod{p})^a \pmod{p}$. V laskee luvun $(g^a \pmod{p})^b \pmod{p}$.

Diffie-Hellmania voidaan käyttää istuntoavaimen luomiseen muodostetun yhteyden ajaksi. Istuntoavaimella voidaan salata viestejä salausalgoritmeja käyttäen.  Asiakas/käyttäjä voi olla palveluntarjoajan tiedossa jonkin aikaa, mutta pidemmän ajan kuluessa avain tulisi vaihtaa tai vastaavasti istunto sulkea. Satunnaislukujen käyttö avaimen luonnissa tekee tietoturvasta paremman. Luonnollisesti avaimen pituuden sekä alkioiden tulee olla suuria, jotta niiden arvaaminen on ulkopuoliselle tunkeutujalle vaikeampaa. \cite{enti} 


\subsection{Mobiilikaupankäynti}

Mobiilikaupankäynnillä tarkoitetaan mobiililaitteella tehtäviä maksutransaktioita tai ostotapahtuman vahvistavia viestejä. Menetelmä on siis osa elektronista kaupankäyntiä, jossa käytetään digitaalisia allekirjoituksia \cite{e-c}. Schwab ja Yang toteavat  \cite{enti} suurten datamäärien varastoinnin olevan yleisiä nykyaikaisilla mobiililaitteilla. Samadanin, Shajarin ja Ahanihan artikkelissa \cite{proxy} esitellään huutokauppasovellus, joka vaatii jokaisen huudon varmistuksen lyhyen ajan sisällä laitteella. Allekirjoitusten luonti tulee olla siis nopeaa mobiililaitteilla tietoturva huomioon ottaen. Sekä laite- että palvelinpohjaisia allekirjoituksia käytetään mobiilikaupankäynnissä \cite{proxy}. Verkkopankki, maksusuoritukset, terveydenhoito ja äänestys ovat mahdollisia kannettavilla laitteilla, mutta langaton verkko tuo ongelmansa kaistanleveyden kanssa \cite{ECC}. 



\section{Laitepohjaiset allekirjoitukset}

Mobiililaite koostuu SIM-kortista ja laitteesta, jossa allekirjoituksen luonti tapahtuu prosessorilla. Laitteen käyttöjärjestelmän on tuettava yleisesti käytettyjä protokollia, jotta salaus, tiviisitefunktiot, varmenteet ja digitaaliset allekirjoitukset ovat mahdollisia. Tietoturvan kannalta SIM-korttia voidaan pitää parempana vaihtoehtona, mutta allekirjoitusten luomisen nopeudessa prosessori on tehokkaampi. Salaisen avaimen säilytyspaikka tulee kuitenkin valita turvallisesti, jotta ulkopuolinen tunkeutuja ei saa tietää salaista avainta. Lisäksi on olemassa malli, jossa SIM-kortti ja laitteen prosessori yhdessä osallistuvat allekirjoituksen luontiin (hybridimalli). Seuraavat alaotsikot perustuvat Samadanin, Shajarin ja Ahanihan malleihin \cite{proxy}. 
 
\subsection{SIM-kortilta luonti}

Laitteen SIM-korttia voidaan pitää turvallisimpana paikkana säilyttää salaista avainta. Edes käyttäjä itse tai laitteen käyttöjärjestelmä ei pääse käsiksi salaiseen avaimeen kortilla. Kuitenkin SIM-kortin laskentakapasiteetti on huomattavasti pienempi kuin laitteen prosessorin. Allekirjoituksen luonti SIM-kortilla on erittäin hidasta.   

\subsection{Laitteen prosessorilla luonti}

Salaisen avaimen säilytys voi tapahtua myös laitteen muistissa. Digitaalinen allekirjoitus luodaan tällöin laitteen prosessorilla, joka on laskentateholtaan huomattavasti tehokkaampi kuin SIM-kortti. Käyttöjärjestelmä voi myös tarjota kirjastoja ja työkaluja allekirjoitusten luontiin. Laitteen käyttöjärjestelmässä voi kuitenkin olla tietoturva-aukko, jota hyväksikäyttäen tunkeutujat voivat saada haltuunsa käyttäjän salaisen avaimen.

\subsection{Hybridimalli}

Hybridimallissa salainen avain joudutaan hetkellisesti paljastamaan laitteen käyttöjärjestelmälle. Tässä menetelmässä on siis olemassa pieni tietoturvariski. Hyvänä puolena hybridimallissa on sen lähes yhtä nopea tehokkuus kuin prosessorilla luonnissa. Monet graafisen käyttöliitymän vaativat ohjelmat tarvitsevat prosessorin laskentatehoa, mutta SIM-kortti voi toimia tietoturvan kannalta avaimen yleisenä säilytyspaikkana. Mallissa allekirjoitus siis luodaan prosessorilla, jolloin salaista avainta käytetään vain hetkellisesti laitteessa.



\subsection{Tunnistautuminen laitteella}
Kun käyttäjä haluaa lähettää viestin palvelimelle tai toiselle käyttäjälle, on tärkeää suosia turvallista protokollaa. On turvallista varmistaa myös oikean henkilön käyttävän laitetta, sillä ulkopuolinen varas on voinut anastaa laitteen. Käyttäjän tunnistautuminen voi perustua salasanan syöttämiseen tai visuaaliseen todennukseen. Istunto laitteen ja palvelimen välille voidaan muodostaa Diffie-Hellmanin protokollaa käyttäen. Viestit salataan yhteisellä avaimella. RSA on kuitenkin parempi välimieshyökkäystä vastaan.

\section{Palvelinpohjaiset allekirjoitukset}

Palvelin voi luoda digitaalisen allekirjoituksen käyttäjän puolesta, kunhan käyttäjä voidaan todentaa palvelimelle. Palvelinten rooli digitaalisten allekirjoitusten luonnissa oli merkittävä aikana, jolloin laitteissa ei ollut tarpeeksi tehoa allekirjoituksen luomiseen. Nykyään laitepohjaiset allekirjoitukset ovat yleistyneet. \cite{proxy}

\subsection{Välityspalvelin}

Välityspalvelin toimii siis eräänlaisena kirjautumispalvelimena käyttäjän ja lopullisen palveluntarjoajan välissä. Välityspalvelin voi luoda allekirjoituksen, mutta oikean käyttäjän todennus vaaditaan. Varmennus voi perustua algoritmeihin kuten RSA tai DSA. On myös mahdollista, että käyttäjälle tehdään varmenne, jolla hän on tunnistettavissa jatkossa palvelimelle \cite{proxy}.

\subsection{NRS ja NRR}

Kiistämättömyys on olennainen osa digitaalista allekirjoitusta. NRS (Non-Repudation of Sender) tarkoittaa, lähettäjä ei voi jälkikäteen kiistää lähettäneensä viestin. NRR (Non-Repudation of Receiver) puolestaan merkitsee vastaanottajan kiistämättömyyttä. Tiivistefunktioilla varmistetaan datan eheys kuten esimerkiksi MD5:llä. Sekä lähettäjän että vastaanottajan on luotava julkiset avaimet ja merkit kirjautumispalvelimelle tunnistettavaksi. Kirjautumispalvelin pyytää varmenteen varmenneviranomaiselta ja muodostaa oman varmenteen lähettäjälle. Näin ollen kirjautumispalvelin voi jatkossa toimia pysyvämpänä vahvistajana lähettäjän ja vastaanottajan välillä. \cite{gene}

\subsection{Yhdistetty allekirjoitus}

Laitteella pystyy delegoimaan allekirjoituksen luonnin palvelimelle kokonaan, osittain tai valtakirjalla. Välityspalvelin voi kokonaan luoda allekirjoituksen käyttäjän salaisella avaimella. Tämä tyyppi ei ole tietoturvan kannalta suotavaa. Osittaisessa allekirjoituksessa käyttäjä luo omasta salaisesta avaimestaan välityspalvelimelle uuden avaimen. Välityspalvelimella on tällöin mahdollisuus tehdä allekirjoitus käyttäjän puolesta. Kiistattomuus nousee näissä kahdessa menetelmässä ongelmaksi. Vastaanottaja ei voi tietää, onko allekirjoitus tullut välityspalvelimelta vai käyttäjältä. Kolmas menetelmä on valtakirjan luovuttaminen välityspalvelimelle. Käyttäjä siis kertoo valtakirjallaan luovuttaneensa allekirjoitusoikeuden toiselle palvelimelle. Valtakirja luodaan käyttäjän salaisella avaimella. Valtakirjan laatiminen saattaa viedä huomattavasti aikaa ja paljon laskentatehoa. \cite{joint}


\subsection{Varmenteet}

Varmenteet ovat kolmannen osapuolen antaman varmenneviranomaisen todistuksia. Myös välityspalvelin voi luoda varmenteen käyttäjälle \cite{proxy}. Jokin käyttäjä, välityspalvelin tai lopullinen palveluntarjoaja tarvitsee varmenteen jatkuvaa yhteydenpitoa varten, koska varmenne kuuluu digitaalisen allekirjoituksen protokollaan. Varmenne voi olla voimassa päiviä, kuukausia tai vuosia, mutta tietoturvan kannalta varmenteiden ei tulisi olla ikuisia. Varmennetta voidaan pitää luotettavana, jos sen tarjoaa ulkopuolinen varmenneviranomainen. Digitaalinen allekirjoitus vaatii toimiakseen aina varmenteen, mutta varmenne voi toimia irrallisena digitaalisesta allekirjoituksesta esimerkiksi palvelinten välisessä tunnistuksessa \cite{proxy}. PKI-protokollan avulla varmenne voidaan luoda luovuttamalla julkinen avain varmenneviranomaiselle ja lähettämällä varmennepyyntö. Tämän jälkeen käyttäjä vahvistaa vielä itsensä salaamalla viestinsä salaisella avaimellaan. Varmenneviranomainen vastaa luovuttamalla varmenteen käyttäjälle. Varmenteeseen on yleensä merkitty seuraavat tiedot: voimassaoloaika, sarjanumero, versio ja käyttäjän tunniste \cite{ECC}. Vastaanottajan tulee siis ottaa huomioon vanhentunut varmenne. Koska varmenne on yksilökohtainen, hyökkääjä ei tee varastetulla varmenteella mitään. 

\section{Vertailu}

Tehokkuus ja tietoturva ovat tärkeitä ominaisuuksia koskien digitaalisia allekirjoituksia. Vaikka nämä kaksi seikkaa eivät ole suoraan toisensa poissulkevia, on syytä ottaa huomioon kummankin prioriteetti. Erityisesti mobiililaitteilla tehokkuudesta joudutaan yleensä karsimaan, joten valitaan vähemmän tehokas allekirjoitusalgoritmi. Tällöin allekirjoittaminen on hidas prosessi \cite{proxy}.

\subsection{Tietoturva}

Mobiililaitteilla voidaan havaita seuraavia tietoturvariskejä: urkinta, välimieshyökkäys, datan muuntaminen, toisena osapuolena esiintyminen ja laitteen kadottaminen. Urkinnalla tarkoitetaan viestien kuuntelua, mutta se voidaan torjua helposti viestin salakirjoituksella esimerkiksi väliaikaisella istuntoavaimella. Välimieshyökkäys tarkoittaa kolmannen osapuolen asettumista lähettävän ja vastaanottavan osapuolten väliin. Diffie-Hellmanissa piilee tämä riski mutta ei yleensä RSA:ssa \cite{enti}. Datan muuntaminen voidaan estää salakirjoituksella sekä käyttämällä tiivistefunktioita. Toisena osapuolena tekeytyminen ja laitteen kadottaminen voidaan estää salasanan kirjoittamisella laitteelle tai visuaalisella todennuksella. 

Julkisen avaimen infrastruktuuri eli PKI-malli toimii, jos salainen avain säilyy suojassa. Mikäli on pienikin riski, että salainen avain on jokun muun tiedossa tulee avainpari vaihtaa heti. Niin kauan kun diskreetin logaritmin ongelmaa ei pystytä ratkaisemaan järkevässä ajassa, ovat RSA ja Diffie-Hellman turvallisia protokollia. Tiivistefunktioiden tulee olla myös ajan tasalla, jotta allekirjoitusten salaus toimii. Esimerkiksi tulevaa SHA-3 standardia kehitellään paremmaksi tulevaa käyttöönottoa varten \cite{nist}.
	  

\subsection{Tehokkuus}

Suorituskyky on parantunut vuosien saatossa niin tietokoneilla kuin mobiililaitteilla. Prosessorien teknologia on kehittynyt mahdollistaen tiheämmät kellopulssit ja moniydinsuorituksen. Myös tietoliikennenopeuksien kasvamisella on ollut suuri merkitys digitaalisten allekirjoitusten luonnissa. Tehokkuutta tarvitaan nopeisiin allekirjoituksiin lyhyellä aikavälillä. Artikkelissa Self-Proxy Mobile Signature \cite{proxy} esitelty huutokauppasovellus tarvitsee jokaiselle huudolle uuden  allekirjoituksen lyhyen ajan sisällä. Tietoturvasta on tässä tapauksessa erittäin vaikea tinkiä, joten käyttäjän olisi hyvä luoda allekirjoitus omalta laitteeltaan. Tehokkuudessa tulee ottaa huomioon siis salauksen nopeus, tiivisteen luominen ja varmenteen hankinta \cite{proxy}. Luonnollisesti myös palvelinpuolella esimerkiksi klusterointi on luonut mahdollisuuden tehokkaaseen allekirjoitusten/varmenteiden luomiseen monelle käyttäjälle samaan aikaan.

\subsection{Nykyaikaisten menetelmien käyttö}

Laitepohjaiset allekirjoitukset ovat vakiintuneet kokoajan mobiililaitteiden laskentatehon kasvun ansiosta. RSA:n lisäksi elliptiset käyrät ovat yleistyneet niiden paremman tietoturvan ansiosta suhteessa avainten pituuteen bitteinä \cite{ECC}. AES algoritmia voidaan pitää murtumattomana, mutta DSA on murrettavissa jo muutaman bittivuodon avulla \cite{gsm}. Elliptisen käyrän DSA:ta käytetään myös mobiililaitteilla \cite{webs}. RSA:n avaimen pituuden on hyvä olla vähintään 1024 bittiä. Elliptisissä käyrissä riittää 160 bittiä tällä hetkellä \cite{ECC}.

Android-käyttöjärjestelmä tukee Javan virtuaalikonetta (JVM). Java käyttää digitaaliseen allekirjoitukseen tarvittavia protokollia, joita tarvitaan monilla mobiililaitteilla nykypäivänä. Bouncy Castle- paketti tarjoaa Javassa monenlaista kryptografisia algoritmeja tiedon salaukseen ja purkamiseen. Javalla myös satunnaisten olioiden luominen on helppoa. \cite{enti}   

\section{Yhteenveto}  

Tässä tekstissä olemme tarkastelleet digitaalisia allekirjoituksia mobiiliympäristöissä ja mobiililaitteissa. Digitaalisen allekirjoituksen ehtoina ovat vastaanottajan todennus, datan eheys ja lähettäjän kiistämättömyys. Menetelmät allekirjoitusten luontiin vastaavat tietokoneilla samanlaisia menetelmiä. Olemme tarkastelleet julkisen avaimen infrastruktuuria, RSA:n ja Diffie-Hellmanin protokollia tarkemmin sekä mobiilikaupankäyntiä. Digitaalisten allekirjoitusten luonti voidaan jakaa kahteen pääryhmään: laite- ja palvelinpohjaisiin allekirjoituksiin. Laiteella allekirjoituksen voi luoda prosessori tai SIM-kortti. Lisäksi hybridimallin olemassaolo tunnetaan. Palvelinpuolella tulee korostua käyttäjän tunnistaminen ja kirjautumispalvelimen merkitys. Kiistattomuuden tulee toimia delegoinnin yhteydessä. Digitaalinen allekirjoitus voidaan delegoida palvelimelle kokonaan, osittain tai valtakirjalla. Varmenteet ja kiistattomuus luovat digitaalisen allekirjoituksen pohjan. Olemme tarkastelleet tekstin lopussa tietoturvan ja tehokkuuden merkitystä digitaalisissa allekirjoituksissa mobiiliympäristö huomioon ottaen. Nykyaikaisiin menetelmiin voimme luetella RSA:n, DSA:n, Diffie-Hellmanin ja elliptisten käyrien algoritmit, joita muun muassa Android-käyttöjärjestelmä tukee.


% --- References ---
%
% bibtex is used to generate the bibliography. The babplain style
% will generate numeric references (e.g. [1]) appropriate for theoretical
% computer science. If you need alphanumeric references (e.g [Tur90]), use
%
% bibliographystyle{babplain-lf}
%
% instead.

\newpage
\bibliographystyle{babalpha-lf}
\bibliography{viitteet}


% --- Appendices ---

% uncomment the following

%\newpage
%\appendix
 
%\section{Esimerkkiliite}

\end{document}